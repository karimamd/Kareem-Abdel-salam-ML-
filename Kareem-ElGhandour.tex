\documentclass[10pt,a4paper]{altacv}

%% AltaCV uses the fontawesome and academicon fonts
%% and packages.
%% See texdoc.net/pkg/fontawecome and http://texdoc.net/pkg/academicons for full list of symbols.
%% When using the "academicons" option,
%% Compile with LuaLaTeX for best results. If you
%% want to use XeLaTeX, you may need to install
%% Academicons.ttf in your operating system's font %% folder.


% Change the page layout if you need to
\geometry{left=1cm,right=9cm,marginparwidth=6.8cm,marginparsep=1.2cm,top=1cm,bottom=1cm}

% Change the font if you want to.

% If using pdflatex:
\usepackage[utf8]{inputenc}
\usepackage[T1]{fontenc}
\usepackage[default]{lato}

% If using xelatex or lualatex:
% \setmainfont{Lato}

% Change the colours if you want to
\definecolor{VividPurple}{HTML}{3E0097}
\definecolor{SlateGrey}{HTML}{2E2E2E}
\definecolor{LightGrey}{HTML}{666666}
\colorlet{heading}{VividPurple}
\colorlet{accent}{VividPurple}
\colorlet{emphasis}{SlateGrey}
\colorlet{body}{LightGrey}

% Change the bullets for itemize and rating marker
% for \cvskill if you want to
\renewcommand{\itemmarker}{{\small\textbullet}}
\renewcommand{\ratingmarker}{\faCircle}

%% sample.bib contains your publications
\addbibresource{sample.bib}

\begin{document}
\name{Kareem Abdel-salam El-Ghandour}
\tagline{Computer Engineer \& Software Developer}
\personalinfo{%
  % Not all of these are required!
  % You can add your own with \printinfo{symbol}{detail}
  \email{karimamd95@gmail.com}
  \phone{+20-106-4599-416}
  \mailaddress{Al Iman Tower ,st.4 in El-nasr st. Smouha}
  \location{Alexandria, EG}
%  \homepage{marissamayr.tumblr.com/}
%  \twitter{twitter.com/karimamd95}
  \linkedin{linkedin.com/in/kareem-a-elghandour}
   \github{github.com/karimamd} 
%   \orcid{orcid.org/0000-0000-0000-0000} % Obviously making this up too. If you want to use this field (and also other academicons symbols), add "academicons" option to \documentclass{altacv}
}

%% Make the header extend all the way to the right, if you want.
\begin{fullwidth}
\makecvheader
\end{fullwidth}

%% Depending on your tastes, you may want to make fonts of itemize environments slightly smaller
\AtBeginEnvironment{itemize}{\small}

%% Provide the file name containing the sidebar contents as an optional parameter to \cvsection.
%% You can always just use \marginpar{...} if you do
%% not need to align the top of the contents to any
%% \cvsection title in the "main" bar.
\cvsection[page1sidebar]{Technical Experience}

\cvevent{Machine Learning Applications Intern}{Computer Vision Lab at AlexU}{July 2018 -- September 2018}{Alexandria, EG}
\begin{itemize}
\item Succeeded to re-implement an Image Captioning solution using Flickr 8K dataset
\item Adapted with limited resources and used only freely available GPU resources to engineer a working model
\item Utilized a merged  model of pre-trained CNN models and LSTMs
\item Reported results,reasons for design decisions and points to improve in future work 
\item Designed progress and flow diagrams to aid the team  visualize checkpoints,debug and also for reporting
\item \textit{\textbf{RNNs (LSTM), CNNs, Keras, NLP, Transfer Learning}}
\end{itemize}

\divider



\cvsection{Education}
\cvevent{B.Sc.\ in Computer and Communications Engineering}{Alexandria University}{Sept 2014 -- June 2019 }{}
\textbf{CGPA: 3.35}
\divider

\cvsection{Coursework}
\cvevent{Computer and Communications Engineering Basics and Some of Elective Courses}{Alexandria University}{Sept 2014 -- June 2019 }{}

\begin{itemize}
\item \textbf{Computer Vision, Pattern Recognition, Optimization Techniques}
\item \textbf{Distributed Systems, Computer Networks, Operating Systems}
\item \textbf{Algorithms and Data structures, Databases, OOP}}
\item \textbf{ Computer Architecture, Embedded Systems}
\end{itemize}



\cvsection{Programming Languages}
\cvskill{Java}{4}
% \divider
\cvskill{Python}{4}
% \divider
\cvskill{C}{3}
% \divider
\cvskill{Android}{2}
\cvskill{PHP}{2}

\cvsection{Military Service}
\cvtag{\textbf{Exempted}}
\clearpage


\cvsection[page2sidebar]{More Projects}
\cvevent{Stanford's X-Ray Bone Abnormality Recognition Challenge}{Final project of Computer Vision course}{October 2018}{Alexandria, EG}
\begin{itemize}
\item Succeeded to implement a well documented Abnormality detection model on MURA dataset for 7 different bone types with good results
\item Performed data augmentation and weighted loss function to deal with imbalanced data and produce more training samples
\item Engineered different types of features extracted from data to get better results
\item Designed an Ensemble model of several weaker models and used minimal resources through efficient data generators
\item \textit{\textbf{Keras,Ensemble, Feature Engineering,Transfer Learning, EDA }}
\end{itemize}

\divider
\cvevent{Social Media Website}{Alexandria University}{February 2017}{Alexandria, EG}
\begin{itemize}
\item Implemented a Social media website similar to Facebook which had a main news feed, adding friends, profiles ..etc
\item \textit{\textbf{Native PHP, MYSQL , HTML, CSS}}
\end{itemize}
\divider


\cvevent{Assembler for SIC/XE architecture}{Alexandria University}{December 2016}{Alexandria, EG}
\begin{itemize}
\item Implemented a functioning OOP assembler for the SIC-XE architecture that takes assembly code as input and produces object code and headers
\item Used text manipulation techniques to check for compilation errors and wrong syntax  and output was machine code corresponding to input assembly
\item \textit{\textbf{Java, OOP}}
\end{itemize}
\divider
\cvevent{Android Language Learning App}{Udacity Android Basics Specialization}{November 2017}{Alexandria, EG}
\begin{itemize}
\item Implemented a language learning application that had words in an old language and its English translation with voice pronunciation
\item used Recycler Views for lists, themes and visual polish, image and voice resources and other Android basics
\item \textit{\textbf{Java, Android}}
\end{itemize}
\divider
\cvevent{Lexical Analyzer Generator (Compiler: Phase 1)}{Alexandria University}{April 2019}{Alexandria, EG}
\begin{itemize}
\item Implemented a functioning OOP Lexical Analyzer Generator that reads a programming language's grammar and translates it into a state machine that labels every token in an input program
\item Wrote well structured OOP code
\item \textit{\textbf{Python, OOP, Design Patterns}}
\end{itemize}





\end{document}
\cvevent{Modulation Recognition}{Alexandria University}{May 2018}{Alexandria, EG}
\begin{itemize}
\item Succeeded to implement a modulation recognition model that takes an input time series signal and outputs a predicted modulation technique that was used on it
\item used raw time series data as input features and also engineered some features like taking the gradient, integration and combination of features and processed them with a CNN model
\item \textit{\textbf{Deep Learning, Keras, Python , Jupyter Notebooks, Feature Engineering }}
\end{itemize}

\divider

%\cvsection[page2sidebar]{Volunteering Experience}


% \cvevent{Product Engineer}{Google}{23 June 1999 -- 2001}{Palo Alto, CA}

% \begin{itemize}
% \item Joined the company as employe \#20 and female employee \#1
% \item Developed targeted advertisement in order to use user's search queries and show them related ads
% \end{itemize}






\clearpage


\end{document}
